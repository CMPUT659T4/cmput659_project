\documentclass{article} % For LaTeX2e
\usepackage{nips15submit_e,times}
\usepackage{hyperref}
\usepackage{url}
\title{Using fMRI to Diagnose Schizophrenia}
\author{%add name 
	\\
	Department of Computing Science\\
	University of Alberta\\
	\texttt{} 
\And 
 \\
Department of Computing Science\\
University of Alberta\\
\texttt{}
\And 
\\
Department of Computing Science\\
University of Alberta\\
\texttt{}  
}
\newcommand{\fix}{\marginpar{FIX}}
\newcommand{\new}{\marginpar{NEW}}

\nipsfinalcopy 

\begin{document}
	\maketitle
	\begin{abstract}
		      
	\end{abstract}
	\section{Introduction}
	Diagnosis of schizophrenia is a challenging task that yet to be addressed\cite{McGuire200891}. Although, in recent years methods which use Functional Magnetic Resonance Imaging(fMRI) for mental disorder diagnosis has become more popular, but in case of schizophrenia it still needs to become more robust and reliable. In similar studies\cite{Rish_2013}\cite{Rosa_2013} have been shown that fMRI can be used in conjunction with Sparse Gaussian Markov Random Field to produce high accuracy in diagnosis of illness. However having a dataset with homogeneous distribution of illness makes this result less reliable and creates the need for more evidence using heterogeneous dataset in terms of illness. 
	
	\bibliographystyle{plain}
	\bibliography{T4_Report}
	
\end{document}