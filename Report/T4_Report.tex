\documentclass{article} % For LaTeX2e
\usepackage{nips15submit_e,times}
\usepackage{hyperref}
\usepackage{url}
\usepackage{verbatim}
\usepackage{amsmath}

\title{Using fMRI to Diagnose Schizophrenia}
\author{%add name 
	\\
	Department of Computing Science\\
	University of Alberta\\
	\texttt{} 
\And 
 \\
Department of Computing Science\\
University of Alberta\\
\texttt{}
\And 
\\
Department of Computing Science\\
University of Alberta\\
\texttt{}  
}
\newcommand{\fix}{\marginpar{FIX}}
\newcommand{\new}{\marginpar{NEW}}

\nipsfinalcopy 

\begin{document}

	\maketitle

\begin{abstract}
Diagnosis of schizophrenia is a challenging task for which diagnostic tests
have yet to be developed~\cite{McGuire200891}. Although Functional Magnetic 
Resonance Imaging (fMRI) methods have become more common in the diagnosis of 
mental disorders have become more popular, for schizophrenia diagnosis fMRI 
methods need to be more robust and reliable. Similar 
studies~\cite{Rish_2013}\cite{Rosa_2013} have shown that fMRI can be used in 
conjunction with Sparse Gaussian Markov Random Field (SGMRF) to produce high 
accuracy in diagnosis of illness.
However having a dataset with homogeneous distribution of illness makes this 
result less reliable and creates the need for more evidence using 
heterogeneous dataset in terms of illness. 
In this work we pursue two paths to tackle this problem. First, we evaluate 
performance of Sparse Gaussian Markov Random Field (SGMRF) on fMRI data 
brain scans, and second we study on Regions of Interest (ROI) as defined by
Power \emph{et al.}~\cite{Power_2011}. 
We used 5 fold cross validation for hyper parameter tuning and $20\%$ holdout 
set for test. Accuracies that We have obtained the following accuracies using 
this method: —— for whole brain features and —- for ROI features. While these 
result are slightly less than the results obtained by Rish \emph{et al.}, 
they are on par with Rosa \emph{et al.} results.  
\end{abstract}


\section{Introduction}
Schizophrenia is a mental/psychiatric disorder~\cite{Rish_2013, Kenji_2010} 
known affect blood flow in the brain~\cite{Kenji_2010} where those who are 
affected can experience hallucinations, delusions and diminished mental 
capacities to varying extents~\cite{jablensky2010diagnostic}. While several
features of schizophrenia have proven useful for its diagnosis there are
current no set of features that have sufficient sensitivity or specificity
to be used in diagnostic tests~\cite{jablensky2010diagnostic}. This 
effectively means that subjectivity plays a role when a physician is 
diagnosing a patient. 

Functional Magnetic Resonance Imaging (fMRI) is a tool for recording 
functional changes caused by neuron activity\cite{}. When a person is doing a 
task, neuron activity fluctuates and in order to provide the energy 
needed for this activity, the blood flow increases to feed the neurons with 
the needed glucose, which is not stored in the brain\cite{}. More blood flow also 
brings more oxygen through blood vessels. This change in the level of 
oxygenated blood known as oxyhemoglobin and deoxyhemoglobin (oxygenated or 
deoxygenated blood) changes the magnetic susceptibility of blood (BOLD signal) 
which is detectable through fMRI~\cite{}.

fMRI is one of the most used and efficient tools in the study of 
psychiatric disorders such as Schizophrenia\cite{}. An advantage of fMRI in
medical diagnosis is that it is non-invasive. This means that unlike some 
other imaging methods, no instruments or dyes are placed in the patient’s body
body, this method operates without using them\cite{}. 

One of the approaches that has been used for studying schizophrenia is 
Sparse Gaussian Markov Random Field(SGMRF)~\cite{Rish_2013}\cite{Rosa_2013}. 
The primary advantage of using this method is that the functional network of 
the brain can be captured using the precision matrix~\cite{Rish_2013}.
By using the resulting network, healthy subjects can be differentiated from 
schizophrenic ones, by observing differences in the functional connectivity 
of the brain. Currently automated approaches to schizophrenia
diagnosis have been able to yeild accuracies of $93\%$ for data that 
originates from a single location~\cite{Rish_2013} and up to approximately 
$80\%$ for data that originates from multiple locations~\cite{Cheng2015}.

In this work we consider

The rest of the paper is organized as follows.


\section{Background and Prior Work}

\subsection{Regions of Interest and Single-Voxel Analysis}
Traditionally in fMRI analysis there are two main approaches for extracting
information from the fMRI image. The first is a single-voxel approach and 
the second is to study regions of interest (ROI)~\cite{heller2006cluster}. 
The tradeoff between these two approaches is that a single-voxel approach
requires the analysis of every voxel and is subject to the low signal to 
noise ratios of individual voxels, whereas a region based approach is only
as effective as the regions are relevant to the fMRI 
task~\cite{heller2006cluster}. In 2011, Power \emph{et al.} identified 264 
putative function regions of interest derived from resting state fMRI, where 
no specific task being performed during data collection~\cite{Power_2011}. 
These regions contain substructures that agree that with known functional 
brain systems and therefore can be seen as fairly accurate 
representations~\cite{Power_2011}.

\subsubsection{Calculating Degrees}
% Double check with Rish paper, this needs work.
When analyzing fMRI data features such as voxel degrees can be extracted for
use with a machine learner. Voxel degrees represent the connectedness of
voxels in the brain with the other voxels and are described as ``the number
of voxel neighbours in a network''~\cite{Rish_2013}. Degrees are calculated 
by performing
multiple Pearson correlation comparisons between the $i^{th}$ voxel and every
other voxel. Once correlation values have be determined, a threshold is
applied to the correlation matrix. This results in binary matrix where 1 
represents a correlation value above the threshold and 0 represents a value 
below. Finally, for each voxel the number of 1 entries are summed (excluding 
the comparison against itself) and this becomes the degree of the voxel.

%\subsection{Fourier Transform}
%A Fourier transform allows for the translation of any signal from the time
%domain to the frequency domain. More specifically, it takes a signal and
%decomposes it into a series constituent $\sin$ and $\cos$ componets. When 
%taking the Fast Fourier Transform or FFT of real data the resulting peaks in 
%the frequency domain are conjugate symmetric~\cite{duhamel1990fast}, meaning 
%that methods that only require the real component of the data need only work 
%with half of the resulting component coefficients in the transform. These 
%Fourier coefficients can be used in place of the original signal as features
%provided to a machine learning classifier.

\subsection{Multi-site Comparisons}
% Mario please add info on Chang

\subsection{Principal Component Analysis}

\subsection{Support Vector Machines}
What is an SVM and how does it work

\subsection{SGMRF}
What is a SGMRF and how does it work

One variation of Markov Random Field is Gaussian Random which is mostly being used for continuous space of variables and has well-defined mathematic properties that can be computed. Multivariate Gaussian density function over set of random variables X is defined as below: 
\begin{equation}\label{gaussian1}
p(X) = (2\pi)^{-n/2} |\Sigma|^{-1/2} \exp\left\{ -\frac{1}{2}(X - \mu)^t \Sigma^{-1} (X - \mu) \right\}.
\end{equation}
 
Where $ \mu $   is mean and $ \Sigma $ is the covariance matrix. We can set the $ \mu$ to zero and replace $\Sigma^{-1}$ with $C$ the equation \eqref{gaussian1} can be written as the following form.
\begin{equation}\label{gaussian2}
p(X) = (2\pi)^{-n/2} |C|^{1/2} \exp\left\{ -\frac{1}{2}(X - \mu)^t \Sigma^{-1} (X - \mu) \right\}.
\end{equation}
\section{Methodology}

\subsection{Data Set}
Turn our slide into this section
How do we have? Balance? 
How we create the holdout set?


\subsection{Your Approach? Mario}
First, we ran the approach proposed by Rish \emph{et al.}~\cite{Rish_2013} on our data set. Here, we obtained the log degrees of each voxel for each subject. Thus we had a matrix of size  $380 \times 28720$. However, using all the available voxels resulted in computation difficulties and thus selecting the best voxels to separate the healthy and schizophrenic subjects is very important. 

To select the best voxels out of the possible $28720$ voxels, we experimented multiple approaches. These include \begin{enumerate}
  \item Selecting the voxels based on the t-test.
  \item Selecting the voxels based on the absolute differences between the mean degrees of a voxel between schizophrenic and healthy subjects.
  \item Selecting the voxels based on the differences between the mean degrees of a voxel between schizophrenic and healthy subjects.
  \item Selecting the voxels based on the differences between the mean degrees of a voxel between healthy and schizophrenic subjects.
\end{enumerate}
Out of the above four approaches we obtained the highest accuracy on our cross validation set by the third approach. Also the number of voxels to select $k$ was decided by running the same approach for different $k$ values. And we found the optimum $k$ value as $20$ for our cross validation set. By using the selected voxels we obtained the precision matrices for schizophrenic and healthy subjects. Similar to the value of $k$, the sparsity coefficient $\lambda$ was also obtained through a hyper parameter serach on our cross validation set.

\subsection{Methods using Power \emph{et al.}'s ROIs}

\subsubsection{ROI with Patient Concatenation}

\subsubsection{ROI with Fouier Coefficients}

\subsubsection{Region Degrees and SVMs}

\subsubsection{Individual MRF Structure Classification}

\subsection{Your Approach? Farhad}
Describe your experiments

\section{Results}

\subsection{Your Approach? Mario}
With the Rish et al. code we obtained a $ 69\%$ cross validation accuracy on our cross validation sets and $75 \%$ accuracy on the hold-out set.

\subsection{Methods using Power \emph{et al.}'s ROIs}

\subsubsection{ROI with Patient Concatenation}

\subsubsection{ROI with Fouier Coefficients}

\subsubsection{Region Degrees and SVMs}

\subsubsection{Individual MRF Structure Classification}

\subsection{Your Approach? Farhad}
Report your results

\section{Conclusions}


\bibliographystyle{plain}
\bibliography{T4_Report}

	
\end{document}
